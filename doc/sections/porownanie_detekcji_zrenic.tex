\newpage
\section{Porównanie detekcji źrenic}


\subsection{Testowanie na statycznych zdjęciach}

\subsubsection{Oczekiwany wynik}

\subsubsection{Zbierane dane}

Podczas testów zbierane były następujące dane:

\begin{itemize}
    \item \textbf{W obszarze tęczówki} - ilość wykrytych punktów znajdujących się wewnątrz tęczówki oka
    \item \textbf{Poza obszarem tęczówki} - ilość wykrytych punktów znajdujących się poza tęczówką oka
    \item \textbf{Średni błąd} - średni błąd ze wszystkich detekcji (patrz Uwaga 1.)
    \item \textbf{Błąd <= 0.1} - ilość detekcji z~błędem nie większym niż $10\%$
    \item \textbf{Błąd <= 0.05} - ilość detekcji z~błędem nie większym niż $5\%$
\end{itemize}

\par

\textit{Uwaga 1.} Błąd dla danego zdjęcia obliczany był następującym wzorem

\begin{align}
    b(P_1, P_2) = \frac{dist(P_1, P_2)}{hypot(w, h)} * 100\%
\end{align}

Gdzie:

\begin{itemize}
    \item \textbf{P\textsubscript{1}} - wykryty punkt
    \item \textbf{P\textsubscript{2}} - oczekiwany środek źrenicy
    \item \textbf{dist} - odległość między punktami (euklidesowa)
    \item \textbf{hypot} - przeciwprostokątna dla podanych przyprostokątnych (w przypadku regionu oka jest to przekątna)
    \item \textbf{w}, \textbf{h} - szerokość i~wysokość regionu oka
\end{itemize}



\subsubsection{Wybór funkcji projekcji}

Ze względu, że metoda PF może wykorzystywać różne funkcje projekcji do swojego działania konieczne było wyznaczenie i~wybranie najlepszej opcji. Badanie skuteczności przeprowadzone było dla następującego zestawu funkcji:

\begin{itemize}
    \item Całkowa
    \item Wariancja kwadratowa
    \item Wariancja pierwiastkowa
    \item Wariancja liniowa
    \item Ogólna z~wariancją kwadratową
    \item Ogólna z~wariancją pierwiastkową
    \item Ogólna z~wariancją liniową
\end{itemize}

W~przypadku funkcji ogólnej dodatkowo zostało przeprowadzone przeszukiwanie celem ustalenia najlepszego współczynnika $\alpha$ dla poszczególnych opcji. Badana była wielkość tego parametru w~przedziale $[0.01, 0.99]$ ze skokiem $0.01$ (wykluczono wartości $0.00$ oraz $1.00$ bo odpowiadają one odpowiednio funkcji całkowej i~funkcji wariancji).

\par

Finalnie najlepsze w~poszczególnych wariantach okazały się następujące wartości współczynnika $\alpha$:

\begin{itemize}
    \item Ogólna z~wariancją kwadratową: $0.99$
    \item Ogólna z~wariancją pierwiastkową: $0.01$
    \item Ogólna z~wariancją liniową: $0.31$
\end{itemize}

W przypadku funkcji ogólnej z~wariancją kwadratową najlepszy rezultat dał parametr $\alpha=0.99$, co w~przybliżeniu degeneruje ją do wariancji kwadratowej. Natomiast w~przypadku pierwiastkowej ($\alpha=0.01$) do funkcji całkowej. Potwierdzają to również wyniki w~tabeli~\ref{tab:eye_pupil_projection_accuracy_result}, gdzie opisane przypadki osiągają takie same lub zbliżone rezultaty.

\begin{table}[!h]

\centering
\caption{Skuteczność algorytmu PF zależnie od funkcji projekcji na zbiorze danych}
\label{tab:eye_pupil_projection_accuracy_result}
\begin{tabular}{|c|c|c|c|c|c|}
\hline
 &
  \textbf{\begin{tabular}[c]{@{}c@{}}W obszarze \\ tęczówki \end{tabular}} &
  \textbf{\begin{tabular}[c]{@{}c@{}}Poza obszarem \\ tęczówki \end{tabular}} &
  \textbf{\begin{tabular}[c]{@{}c@{}}Średni \\ błąd\end{tabular}} &
  \textbf{\begin{tabular}[c]{@{}c@{}}Błąd \\ <= 0.1\end{tabular}} &
  \textbf{\begin{tabular}[c]{@{}c@{}}Błąd \\ <= 0.5\end{tabular}}
  \\ \hline \hline

\textbf{Całkowa} &
  85 &
  51 &
  14,35\% &
  53 &
  26  \\ \hline
  
\textbf{Wariancji, kwadratowa} &
  98 &
  38 &
  13,51\% &
  60 &
  27 
  \\ \hline
  
\textbf{Wariancji, pierwiastkowa} &
  79 &
  57 &
  14,66\% &
  47 &
  29  \\ \hline
  
  \textbf{Wariancji, liniowa} &
  51 &
  85 &
  23,12\% &
  24 &
  4  \\ \hline
  
  \textbf{Ogólna, kwadratowa} &
  98 &
  38 &
  13,51\% &
  60 &
  27 
  \\ \hline
  
\textbf{Ogólna, pierwiastkowa} &
  85 &
  51 &
  14,27\% &
  53 &
  26  \\ \hline
  
  \textbf{Ogólna, liniowa} &
  86 &
  50 &
  14,17\% &
  53 &
  26  \\ \hline
  
  \hline
\end{tabular}%
\end{table}

Najlepszą skuteczność wykazała funkcja wariancji kwadratowej oraz ogólna kwadratowa. Wykryły ona najwięcej punktów wewnątrz tęczówki - $72\%$ oraz uzyskały najmniejszy średni błąd - $13.51\%$. Ze względu na opisaną wyżej degradacje, której przyczyną jest współczynnik $\alpha$ z~tej pary wybrana została funkcja wariancji, która musi wykonać mniej obliczeń w~czasie swojego działania. Dlatego ta wersja funkcji była wykorzystywana w~dalszej części badań.

\subsubsection{Badanie skuteczności detekcji}






\begin{table}[!h]
\label{tab:eye_pupil_detection_accuracy_result}
\centering
\caption{Skuteczność algorytmów detekcji źrenic na zbiorze danych}
\begin{tabular}{|c|c|c|c|c|c|}
\hline
 &
  \textbf{\begin{tabular}[c]{@{}c@{}}W obszarze \\ tęczówki \end{tabular}} &
  \textbf{\begin{tabular}[c]{@{}c@{}}Poza obszarem \\ tęczówki \end{tabular}} &
  \textbf{\begin{tabular}[c]{@{}c@{}}Średni \\ błąd\end{tabular}} &
  \textbf{\begin{tabular}[c]{@{}c@{}}Błąd \\ <= 0.1\end{tabular}} &
  \textbf{\begin{tabular}[c]{@{}c@{}}Błąd \\ <= 0.5\end{tabular}}
  \\ \hline \hline

\textbf{CDF} &
  113 &
  23 &
  10,72\% &
  87 &
  54  \\ \hline
  
\textbf{PF} &
  98 &
  38 &
  13,51\% &
  60 &
  27 
  \\ \hline
  
\textbf{EA} &
  68 &
  68 &
  19,68\% &
  42 &
  14  \\ \hline
  
  \hline
\end{tabular}%
\end{table}


W przypadku progowania przy pomocy dystrybuanty na prawidłową detekcje wpływ miała jakość wycinka zdjęcia podawanego na wejście metody. w~przypadku gdy duża była jego ostrość, a~oko zajmowało subiektywnie duży obszar to algorytm radził sobie bardzo dobrze. Metoda uzyskiwała dobre wyniki zarówno w~przypadku gdy tęczówka oka była ciemna, jak i~jasna. Negatywny wpływ na skuteczność detekcji miało występowanie na wycinku zdjęcia brwi, a~w~szczególności gdy były one w~ciemnym odcieniu lub nałożony był na nie mocny makijaż. Podobnie intensywna barwa rzęs skutkowała obniżeniem dokładności wskazywania lokalizacji źrenicy. Lepsze rezultaty metoda uzyskiwała gdy na wycinku znajdowała się tylko i~wyłącznie gałka oczna. w~przypadku gdy źrenica miała jasny odcień wynikający np. z~dużego natężenia padającego światła to uzyskiwane rezultaty nie były zadowalające. Jeśli algorytmowi udało się wskazać punkt należący do tęczówki, to w~większości przypadków wynikiem był jej środek oraz źrenica. w~ogólności, skuteczność tej metody opierała się w~dużej mierze na występaniu ciemnych punktów innych niż tęczówka i~źrenica.

\par

Metoda projekcji lepsze rezultaty detekcji wykazywała na obszarach oczu o~małym kontraście i~ostrości. Gorsza jakość miała pozytywny wpływ na jej działanie. Częstym zjawiskiem było wskazanie tylko jednej składowej poprawnie - poziomej lub pionowej. Algorytm ten w~części przypadków wskazywał rzęsy zamiast źrenic. Prawidłowe detekcji występowały przy bardzo wyraźnych przejściach między białą częścią oka, a~tęczówką. Jeśli region źrenicy był ciemniejszy niż reszta obszaru oka to detekcja uzyskiwała lepsze rezultaty. Występowanie ramek okularów przeszkadzały w~prawidłowym wskazaniu szukanego punktu.

\par

Analiza krawędzi całkowicie nie radziła sobie z~detekcją w~przypadku gdy występowało dużo wyraźnych konturów innych niż tęczówka lub źrenica. Dobre rezultaty osiągane były jeśli obszar oka był jasny i~naświetlony. w~przeciwieństwie do projekcji metoda ta lepiej radziła sobie przy wycinkach dobrej jakości i~o dużej ostrości. Porównując ten algorytm do dwóch pozostałych jego zachowanie wydawało się dużo bardziej losowe, ponieważ w~wielu przypadkach trudno było wskazać powody zwróconej złej lokalizacji. Wyniki procentowe jak i~subiektywne odczucia wskazują jednoznacznie, że metoda oparta na analizie krawędzi radziła sobie najgorzej.





\subsubsection{Badanie szybkości detekcji} \label{section:test_pupil_speed_img}

\begin{table}[!h]
\label{tab:eye_pupil_speed}
\centering
\caption{Czas przetwarzania algorytmów detekcji źrenic na zbiorze danych.}

\begin{tabular}{|c|c|c|c|}
\hline
 & 
  \textbf{\begin{tabular}[c]{@{}c@{}}Całkowity czas \\ przetwarzania \end{tabular}} &
  \textbf{\begin{tabular}[c]{@{}c@{}}Średni czas\\ przetwarzania \\ pojedynczej iteracji\end{tabular}} &
  \textbf{\begin{tabular}[c]{@{}c@{}}Średni czas\\przetwarzania \\ pojedynczego\\zdjęcia\end{tabular}} \\ \hline\hline
  
\textbf{PF} & 
  0,272 s &
  0,00272 s &
  0,00002001 s    \\ \hline
  
\textbf{EA} & 
  0,765 s &
  0,00765 s &
  0,00005627 s  \\ \hline
  
\textbf{CDF} & 
  0,637 s &
  0,00637 s &
  0,00004684 s \\ \hline

  \hline
\end{tabular}%

\end{table}

Najszybszym okazał się algorytm oparty na funkcji projekcji. Przetworzenie jednego wycinka oka zajęło mu średnio $2.0*10^{-5}s$. Dwukrotnie dłużej trwało wykonanie metody z~użyciem progowania opartego o~dystrybuantę. Najwolniej natomiast działała analiza krawędzi uzyskując czas $5.6*10^{-5}$.

\par

Wyniki zdają się oddawać naturę poszczególnych metod, ponieważ projekcja wymaga maksymalnie dwóch przejść całej macierzy obrazu. Natomiast algorytm oparty na analizie krawędzi wykorzystuje filtry oraz \textit{Canny}, dlatego ma największą złożoność czasową.

\par

Wszystkie metody są bardzo szybkie (rząd wielkości $10^{-5}s$) i~nawet najwolniejsza z~nich mogłaby być wykonana prawie 18 tyś. razy na sekundę. Przekłada się to na zerowe obciążenie całej aplikacji, która wykrywając jedynie twarz osiąga raptem $16$ klatek na sekundę dla barw RGB (patrz rozdz.~\ref{section:face_speed_live}). z~tego powodu podczas wyboru najlepszego algorytmu detekcji źrenic nie była brana pod uwagę szybkość poszczególnych metod, a~jedynie ich skuteczność.




\subsection{Testowanie na obrazie z~kamery}

Przetestowanie skuteczności detekcji na obrazie z~kamery i~przedstawienie ich w~postaci liczbowej byłoby bardzo trudne. z~tego powodu algorytm został przebadany poprzez obserwacje zachowania detekcji na podglądzie na żywo. Badania były przeprowadzone w~takich samych warunkach jak pozostałe elementy systemu. Wyniki są subiektywnym odczuciem autora projektu i~przedstawione w~formie opisowej. 

\par

Ze względu na wyniki szybkościowe detekcji źrenic (patrz rozdz.~\ref{section:test_pupil_speed_img}), które jednoznacznie wykazały marginalny czas przetwarzania wszystkich algorytmów, testy czasowe na obrazie z~kamery na żywo zostały uznane za niepotrzebne i~pominięte.

\subsubsection{Badanie skuteczność detekcji}

Podobnie jak w~testach przeprowadzonych na statycznych zdjęciach CDF uzyskał subiektywnie najlepsze rezultaty. We wszystkich scenariuszach sprawdził się wystarczająco. Nawet w~momencie gdy nie wykrywał dobrze środka źrenicy to wskazania znajdowały się w~obszarze tęczówki. Pracował stabilnie zarówno w~centralnym położeniu oka jak i~zwróconym w~bok. 

\par 

Algorytm PF radził sobie dobrze tylko połowicznie. Najlepsze rezultaty osiągnął w~przypadku światła padającego zza użytkownika. Wtedy zarówno centralne jak i~boczne położenie był prawidłowo wykrywane. w~pozostałych badaniach dobre wyniki osiągał gdy oczy był skierowane w~bok, a~w~przypadku spojrzenia wprost przed siebie lokalizacja podawana była chaotycznie. Test w~ciemnym pomieszczeniu potwierdził, że metoda dobrze radzi sobie przy małym kontraście i~jasności. 

\par

Analiza krawędzi w~przypadku obrazu na żywo w~ogóle się nie sprawdziła. w~żadnych warunkach detekcja nie była wystarczająca na potrzeby pracy dyplomowej. Przez większość czasu zwracaną lokalizacją były rogi obszaru oczu.


 

\subsection{Wybór algorytmu detekcji źrenic}

Najskuteczniejsza okazała się metoda oparta na progowaniu z~użyciem dystrybuanty. w~teście na zbiorze danych~$83\%$ wykrytych punktów znajdowało się wewnątrz tęczówki. Subiektywnie, najlepsze i~wystarczające rezultaty uzyskała również w~badaniu obrazu na żywo z~kamery.

\par

Szybkość wszystkich algorytmów stała na tak wysokim i~marginalnym poziomie, że nie była brana jako czynnik wpływający na wybór algorytmu.

\par

Z powodu wykrywalności źrenic rozwiązanie CDF zostało wybrane jako podstawowe i~używany w~projekcie.