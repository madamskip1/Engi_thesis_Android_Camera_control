\newpage

\section{Podsumowanie}

\subsection{Wykonane prace}

Na początku wykonano przegląd dostępnych rozwiązań mających na celu detekcję twarzy i~jej punktów charakterystycznych, oczu oraz źrenic. Część z~nich dostępna była w~bibliotekach OpenCV (wraz z~dodatkowymi modułami) oraz Dlib. Wykorzystując krzyżową kompilację udało się je dostosować do użytku w~projekcie.

\par

Wybrane algorytmy zostały przetestowane pod kątem skuteczności oraz złożoności czasowej. Badania były prowadzone zarówno przetwarzając statyczne zdjęcia, jak i~kolejne klatki obrazu na żywo z~kamery. Dla każdego zdjęcia zostały przygotowane i~opisane ręcznie takie informacje jak: położenie twarzy czy oczu, co pozwoliło na powtarzalne wyniki liczbowe testów. Na podstawie przeprowadzonych badań wybierano najlepsze metody na potrzeby projektu. Okazały się nimi:

\begin{itemize}
    \item do detekcji twarzy - Histogram zorientowanych gradientów +~maszyna wektorów nośnych
    \item do detekcji facemarków - Kazemi
    \item do detekcji oczu - Eye Aspect Ratio
    \item do detekcji źrenic - CDF
\end{itemize}

\par

Niektóre z metod były optymalizowane i~dostrajane z~użyciem autorskich rozwiązań, które pozwoliły na uzyskanie lepszych rezultatów niż surowe implementacje. Zaproponowane zmiany wpływały głównie na skuteczność detekcji.

\par

Finalnie, udało się zrealizować założony cel pracy tworząc aplikację na urządzenia z~systemem Android, która analizując twarz użytkownika reaguje w~zadany sposób na jego gesty. Celem demonstracyjnym przygotowano dwa scenariusze - informowanie użytkownika, że udało się prawidłowo wykryć jego mrugnięcie oraz prostą galerię zdjęć, w~której kolejne obrazy możemy przeglądać poruszając oczami w~odpowiednią stronę. 

\par

Kod aplikacji wraz z elementami prezentacyjnymi jej działanie znajdują się na repozytorium Git autora, do którego opis wraz z odnośnikiem został umieszczony w załączniku~1. 


\subsection{Zdobyta wiedza i wyciągnięte wnioski}


Czas poświęcony na pracę dyplomową pozwolił poznać wiele aspektów zarówno z~dziedziny przetwarzania obrazów, jak i~tworzenia oprogramowania na systemy Android. 

\par

Przystępując do pracy dyplomowej autor nie miał nigdy wcześniej styczności z~wytwarzaniem oprogramowania na ten system operacyjny, co wiązało się z~koniecznością pozyskania wiedzy całkowicie od podstaw w~tym segmencie. Chociaż zdobyte podczas prac nad projektem umiejętności pozwalają już na pisanie aplikacji na ten system operacyjny, to ilość wiedzy, która jest jeszcze do zdobycia jest ogromna.

\par

Integracja wybranych bibliotek ze środowiskiem Android obciążona była wieloma problemami związanymi z~krzyżową kompilacją z ich języków natywnych do Javy. Dodatkowo wymagały one poznania i~wykorzystania komponentu JNI, który pozwolił wprowadzić elementy języka C++ w~projekcie. Praca nad projektem wymagała również poznania i~zaznajomienia się z~możliwościami oferowanymi przez wybrane biblioteki oraz z~ich działaniem. Niewątpliwie są to bardzo popularne rozwiązania na rynku, więc zdobyta wiedza może być wykorzystana podczas dalszego rozwoju autora w~dziedzinie przetwarzania obrazu. 

\par

Mimo, że finalnie udało się uzyskać wystarczające wyniki czasowe na testowanym urządzeniu to można wysnuć tezę, że urządzenia mobilne z~systemem Android nie są jeszcze w pełni gotową platformą na analizę obrazu w czasie rzeczywistym. Wykorzystywany sprzęt w~czasie rozpoczęcia prac był jednym z~najnowocześniejszych i najwydajniejszych modeli dostępnych na rynku. Z~tego powodu można przypuszczać, że na przeciętnym urządzeniu ilość osiąganych klatek na sekundę mogłaby nie być wystarczająca. Dodatkowo bardzo dużym problemem jest brak kompatybilności niektórych rozwiązań z~kartami graficznymi urządzeń mobilnych, czego konsekwencją jest wykonywanie algorytmów przez jednostki obliczeniowe. Konsekwencją tego był brak możliwości wykorzystania np.~sieci konwolucyjnych, które na kartach graficznych działają z~bardzo dużą częstotliwością. Jednakże rynek urządzeń mobilnych rozwija się bardzo szybko i~w~najbliższej przyszłości wydajne modele będą standardem, co pozwoli na powszechne stosowanie analizy obrazu na żywo. Możliwe też, że wykonanie takich algorytmów będzie przeniesione do zyskującej ogromną popularność chmury, a wtedy mocne urządzenia mobilne nie będą nawet warunkiem koniecznym.





\subsection{Możliwość rozwoju i dalszych badań}

Aplikacja i~projekt pozostawia możliwość dalszego rozwoju. Zarówno przez implementację i~porównanie kolejnych algorytmów detekcji opisanych już elementów, jak i~przez dodawanie analizy innych fragmentów twarzy i~nie tylko.

\par

Inne metody mogą okazać się zarówno bardziej skuteczne, jak również i~szybsze. Możliwe jest, że w~zależności od sytuacji różne rozwiązania dadzą znacznie odmienne wyniki. Z~tego względu można pokusić się o~dostosowanie wyboru algorytmu zależnie od warunków, celem osiągnięcia jak najlepszych rezultatów. Dzięki zaproponowanej architekturze oraz zastosowaniu wzorca projektowego \textit{wstrzykiwanie zależności} zmiana jednego algorytmu danej detekcji na inny jest bardzo prosta, co pozwoli na szybkie wykorzystanie nowych rozwiązań.

\par

Detekcję źrenic można spróbować oprzeć na sieciach neuronowych i~uczeniu maszynowym zamiast na klasycznych metodach przetwarzania obrazu. Może się okazać, że kosztem wydajności zyskamy większą skuteczność.

\par

Jednym z~pomysłów na nowe elementy może być wykrywanie uśmiechu i~emocji użytkownika. Aplikacja reagowałaby na zadowolenie czy zmiany humoru. W~dobie powszechnych mediów społecznościowych mogłoby to być wykorzystane celem automatycznego opiniowania zdjęć, na które użytkownik patrzy w zależności od tego czy się uśmiecha lub czy jest radosny.
