\newpage

\section{Porównanie algorytmów detekcji znaczników twarzy}

Podobnie jak w~przypadku detekcji twarzy przeprowadzone zostały testy algorytmów wykrywania facemarków, przedstawionych w~rozdziale \hyperref[section:landmarks]{\ref{section:landmarks}}. Ze względu na specyfikę nanoszenia punktów charakterystycznych oraz ich ilość, trudno jest określić dokładność działania korzystając z~matematycznych i~liczbowych form wyrazu. Z~tego powodu ocena jakości obu algorytmów to subiektywna opinia autora na podstawie obserwacji znaczników obszaru oczu i~ust. Duża uwaga podczas opiniowania poświęcona została dokładności odwzorowaniu punktów w~przypadku przymkniętych lub całkiem zamkniętych oczu, ponieważ ma to istotny wpływ na inne aspekty pracy dyplomowej. Natomiast złożoność czasowa jest już mierzalna i~została wyrażona liczbowo. 




\subsection{Testowanie na statycznych zdjęciach}

Oba algorytmy zostały przetestowane na statycznych zdjęciach z~ze~zbioru danych w~zakresie skuteczności i~szybkości działania.

\subsubsection{Usunięcie części zdjęć ze zbioru danych}

Ze względu na wybór algorytmu \textit{HOG} do detekcji twarzy konieczne okazało się odrzucenie 2~z~80 przygotowanych zdjęć, ponieważ metodzie tej nie udało się wykryć na nich twarzy (patrz rozdz.~\hyperref[{section:skutecznosc_detekcji_twarzy}]{\textit{\ref{section:skutecznosc_detekcji_twarzy}}}).

\subsubsection{Badanie skuteczności detekcji}

Podczas testu zostały zebrane następujące dane:

\begin{itemize}
    \item \textbf{Prawidłowe detekcje} - pokrycie twarzy znacznikami, które uznane zostały za dobre
    \item \textbf{Złe detekcje} - pozostałe, które nie zostały uznane za dobre
    \item \textbf{Detekcje lepsze niż drugiego algorytmu} - który z~dwóch algorytmów poradził sobie lepiej w danym przypadku testowym. 
\end{itemize}

Zebrane dane są całkowicie subiektywnym odczuciem i~inne osoby mogą  mieć odmienną opinię oraz wyniki.

\vspace{5mm}

Oba algorytmy dawały taki sam rezultat zarówno w~skali szarości jak i~w~trzy kanałowym zestawie barw, dlatego tabela wynikowa została uproszczona przez usunięcie takiego podziału.

\begin{table}[!h]
\label{tab:facemarks_accuracy}
\centering
\caption{Skuteczność algorytmów detekcji landmarków}
\begin{tabular}{|c|c|c|c|}
\hline
 &
  \textbf{\begin{tabular}[c]{@{}c@{}}Prawidłowe\\ detekcje\end{tabular}} &
  \textbf{\begin{tabular}[c]{@{}c@{}}Złe\\ detekcje\end{tabular}} &
  \textbf{\begin{tabular}[c]{@{}c@{}}Detekcje lepsze niż\\ drugiego algorytmu\end{tabular}}   \\ \hline \hline
\textbf{LBF} &
  35 &
  42 &
  16 \\ \hline

\textbf{Kazemi} &
  66 &
  12 &
  62 \\ \hline
 
  
  \hline
\end{tabular}%
\end{table}

Zebrane dane pokazują jasno, że model oparty na metodzie \textit{Kazemi} dał zdecydowanie lepsze wyniki niż drugi badany algorytm. W~62 przypadkach testowych pokrycie twarzy facemarkami było subiektywnie dokładniejsze niż w~metodzie \textit{LBF}. Tylko~12~z~78 detekcji uznałem za błędne. Można przyjąć, że jest to wynik co najmniej poprawny.

\par

Kazemi dobrze radził sobie z~obróconymi i~pochylonymi twarzami, natomiast LBF w~takich przypadkach okazywał się mocno niedokładny i~nakładał znaczniki w sposób podobny jak dla twarzy ustawionych pionowo. Oba algorytmy miały problem w przypadku gdy cień padał na obszar oczu, wtedy znaczniki w~tych rejonach były odchylone od prawidłowych pozycji. Metoda LBF miała problem w~przypadku osób z~ciemniejszymi odcieniami skóry, a~także gdy proporcje twarzy były rozciągnięte. Gdy osoba na zdjęciu nosiła okulary nie wpływało to znacząco na dokładność algorytmu Kazemi w przeciwieństwie do drugiej metody, która osiągała wtedy złe rezultaty. Rozwiązanie LBF natomiast radziło sobie lepiej jeśli twarz i~oczy były częściowo zasłonięte. W~obu przypadkach trudne i~intensywne warunki oświetleniowe wpływały negatywnie na dokładność odwzorowania punktów charakterystycznych. 

\par

Metoda LBF myliła się na wielu zdjęciach bardzo mocno, a~punkty były rozłożone chaotycznie i~losowo. W tych przypadkach trudno oszacować powód, ale jest to fakt praktycznie dyskwalifikujący to rozwiązanie. Ma to odwzorowanie również w~tabeli, gdzie mniej niż połowa wyników została uznanych za dobre. 





\subsubsection{Badanie szybkości detekcji}

W~tym teście zostały zebrane i~porównane następujące dane:

\begin{itemize}
    \item \textbf{Całkowity czas przetwarzania} - suma czasów detekcji znaczników dla wszystkich 20 iteracji
    \item \textbf{Średni czas przetwarzania pojedynczej iteracji} - uśredniony czas detekcji znaczników dla pojedynczej iteracji
    \item \textbf{Średni czas przetwarzania jednego zdjęcia} - uśredniony czas detekcji znaczników dla pojedynczego zdjęcia
\end{itemize}

\begin{table}[!h]
\label{tab:facemarks_speed}
\centering
\caption{Czas przetwarzania algorytmów detekcji znaczników twarzy na zbiorze danych}

\begin{tabular}{|c|c|c|c|}
\hline
 & 
  \textbf{\begin{tabular}[c]{@{}c@{}}Całkowity czas \\ przetwarzania \end{tabular}} &
  \textbf{\begin{tabular}[c]{@{}c@{}}Średni czas\\ przetwarzania \\ pojedynczej iteracji\end{tabular}} &
  \textbf{\begin{tabular}[c]{@{}c@{}}Średni czas\\przetwarzania \\ pojedynczego\\zdjęcia\end{tabular}} \\ \hline\hline
  
\textbf{Kazemi RGB} & 
  5,717 s &
  0,285 s &
  0,00366 s    \\ \hline
  
\textbf{LBF RGB} & 
  6,545 s &
  0,327 s &
  0,00419 s  \\ \hline
  
\textbf{Kazemi sk. szaro.} & 
  5,472 s &
  0,273 s &
  0,00351 s     \\ \hline
  
\textbf{LBF sk. szaro.} & 
  6,084 s &
  0,304 s &
  0,00391 s    \\ \hline
  
  \hline
\end{tabular}%

\end{table}

Algorytm Kazemi z~użyciem biblioteki dlib i~języka C++ okazał się szybszy o~ponad $10\%$ od odpowiednika w postaci LBF. Obie metody uzyskały lepszy czas w~teście opartym na obrazach w~skali szarości niż w~RGB o~kilka procent. Istotnym faktem jest, że oba algorytmy potrzebują mało czasu (rząd wielkości $10^{-3} s$) na przetworzenie pojedynczego zdjęcia, dzięki czemu w~warunkach czasu rzeczywistego nie powodują znacznego spadku klatek na sekundę. 

\subsection{Testowanie na obrazie z kamery na żywo}

Kolejnym etapem testowania detekcji punktów charakterystycznych twarzy było wykorzystanie obrazu z~kamery na żywo. Warunki przeprowadzenia eksperymentu zostały określone w rozdz.~\hyperref[{section:face_detection_test_live}]{\ref{section:face_detection_test_live}}.

\subsubsection{Badanie skuteczność detekcji} \label{section:facemark_live_detection}

Ze względu na opisaną wyżej trudność matematycznego wyrażenia skuteczności nakładania znaczników, wyniki porównania zostały przedstawione w formie opisowej i~są subiektywnym odczuciem autora na podstawie obserwacji działania algorytmu na żywo.

\vspace{5mm}

Algorytm Kazemi bez zarzutu poradził sobie w~trzech scenariuszach. Natomiast w~jednym - przy mocnym oświetleniu padającym na obiektyw i~na twarz - występowało wtedy chwilowe niedokładne dopasowanie. Poza tym problemem radził sobie on bardzo dobrze. Ruchy twarzy nie przeszkadzały w prawidłowym ułożeniu znaczników. Punkty były bardzo stabilne, a~podczas sztywnego położenia twarzy nie występowały ich drgania.

\par

Gorsze wyniki uzyskała metoda LBF. Podobnie jak Kazemi miał pewne problemy podczas scenariusza opartego na mocnym oświetleniu. Występowało ciągłe drganie punktów, nawet podczas sztywnego położenia twarzy. Metoda ta oznaczała twarz jako szerszą niż w~rzeczywistości była. Podczas ruchów twarzy algorytm gubił prawidłowe położenie punktów.

\par

W obu przypadkach potwierdzają się problemy z~testów na statycznych zdjęciach, gdzie intensywne oświetlenie negatywnie wpływało na odwzorowanie punktów charakterystycznych. 

\subsubsection{Badanie szybkość detekcji} \label{section:facemark_speed_live}

Test był przeprowadzony używając detekcji twarzy \textit{HOG}, do którego dostarczano obraz w~przestrzeni barw RGB. 

\begin{table}[!h]
\label{tab:facemarks_speed_live}
\centering
\caption{Szybkość algorytmów detekcji znaczników twarzy dla obrazu na żywo z kamery [klatki/s]}
\begin{tabular}{|c|c|c|c|c|c|}
\hline
 \textbf{\begin{tabular}[c]{@{}c@{}}Warunki\end{tabular}} &
  \begin{tabular}[c]{@{}c@{}}1.\end{tabular} &
  \begin{tabular}[c]{@{}c@{}}2.\end{tabular} &
  \begin{tabular}[c]{@{}c@{}}3.\end{tabular} &
  \begin{tabular}[c]{@{}c@{}}4.\end{tabular} &
  \textbf{\begin{tabular}[c]{@{}c@{}}Średnia:\end{tabular}}\\ \hline \hline
\textbf{LBF RGB} &
  16,076 &
  15,960 &
  15,820 &
  16,079 &
  15,983  \\ \hline

\textbf{LBF sk. szaro.} &
  15,959 &
  16,016 &
  15,829 &
  15,793 &
  15,899 \\ \hline
  
\textbf{Kazemi RGB} &
  15,887 &
  15,941 &
  16,077 &
  16,171 &
  16,019  \\ \hline
  
\textbf{Kazemi sk. szaro.} &
  15,747 &
  16,146 &
  15,866 &
  16,096 &
  15,963 \\ \hline

  \hline
\end{tabular}%

\end{table}

Mniejsza ilość klatek w~przypadku testów w skali szarości prawdopodobnie jest związany z dodatkowym narzutem czasowym w~postaci konwersji obrazu z~trójkanałowej barwy na jednokanałową.

\par

Oba algorytmy uzyskały bardzo zbliżone wyniki. Niewiele szybsza okazała się jednak metoda Kazemi. Rezultat jest porównywalny z~testem przeprowadzonym na statycznych zdjęciach.





\subsection{Wybór algorytmu detekcji znaczników twarzy}

\textit{Kazemi} okazał się przede wszystkim dużo skuteczniejszym i~stabilnym algorytmem niż detekcja znaczników oparta na \textit{Local Binary Features}. Dodatkowo jest on około~$10\%$ szybszy. Wszystkie testy wskazują na wyższość \textit{Kazemi} i~z~tych powodów jest on używany w dalszej części pracy dyplomowej i~projektu, jako główny algorytm określenia położenia punktów charakterystycznych. 