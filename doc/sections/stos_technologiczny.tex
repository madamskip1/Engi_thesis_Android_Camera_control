\newpage

\section{Stos technologiczny} \label{section:tech_stack}

\subsection{System operacyjny Android}

Docelowym systemem operacyjnym, na który przygotowywany był projekt jest Android. Opiera się on na~jądrze Linux przystosowanym do~urządzeń mobilnych. Aktualnie stosowany jest też w~wielu innych urządzeniach elektronicznych takich jak telewizory, systemy audi czy nawet komputery pokładowe w samochodach.  

\subsection{Biblioteki}

\subsubsection{OpenCV}

Otwarta biblioteka widzenia komputerowego (ang. Open Source Computer Vision Library) \cite{opencv} - jedna z~najbardziej rozpowszechnionych bibliotek służąca do~przetwarzania obrazu i~widzenia komputerowego. Napisana z~użyciem języka C/C++ co zapewnia jej szybkie działanie i~możliwość wykorzystania niskopoziomowych mechanizmów. Z rozwiązań tego pakietu można korzystać na wielu systemach operacyjnych, a~dzięki wiązaniom do popularnych języków, kod może być pisany również w Javie czy Pythonie. Dzięki zastosowaniu takich mechanizmów sprzętowych jak CUDA \cite{nvidia_cuda} część obliczeń jest przenoszona z~jednostek arytmetycznych na akceleratory graficzne.


\subsubsection{OpenCV-contrib} \label{section:opencv_contrib}

Zbiorcza nazwa dla dodatkowych modułów \cite{opencv_contrib} biblioteki OpenCV. Nie są one zawarte w~wersji stabilnej API głównej pakietu. Jednak są to rozwiązania, które po szeregu testów i~pewnym okresie przejściowym mogą zostać włączone do podstawowego modułu. Znajdują się tam rozwiązania do~rozpoznawania twarzy \cite{opencvcontribface}, sieci konwolucyjnych czy detekcji obiektów. 

\subsubsection{Dlib}

Wieloplatformowy zestaw narzędzi \cite{dlib} napisany z~wykorzystaniem języka C++. Początkowo głównym obszarem zastosowań było uczenie maszynowe, lecz z~czasem zaczęto rozwijać także sektor sieciowy, przetwarzania obrazów czy operacje numeryczne. Rozpowszechniona jest na~licencji otwartego oprogramowania i~stale rozwijana. 

\subsubsection{CameraX}

Jeden z modułów \cite{camerax} tzw.~\textit{Android Jetpack} \cite{android_jetpack}. Biblioteka pozwalająca na~łatwą obsługę kamer zamontowanych w urządzeniu, na~którym uruchamiany jest program. Pozwala na przechwytywanie na~żywo obrazu, a~następnie jego analizę klatka po klatce oraz zapisywanie uzyskanego nagrania wideo i~zdjęć. 


\subsection{Języki programowania}

W projekcie wykorzystywane są dwa języki programowania:

\begin{itemize}
    \item Java - większość projektu napisane jest w tym języku
    \item C++ - wykorzystywany do wołania metod biblioteki Dlib przy pomocy JNI (rozdz. \ref{section:jni})
\end{itemize}

Dodatkowo, szablony aplikacji Android napisane są z~użyciem języka znaczników XML.

\subsection{JNI} \label{section:jni}

Natywny interfejs programistyczny dla języka Java (ang. Java Native Interface) \cite{jni} -  jest to~interfejs pozwalający uruchamiać natywne programy i~biblioteki w innym języku i~przeznaczone na inne systemy w~wirtualnej maszynie Java. W~projekcie pracy dyplomowej pozwala to~wykorzystać bibliotekę Dlib, która nie posiada gotowych wiązań do Javy. W~tym przypadku kod wywołujący metody tego pakietu jest napisany w~języku C++, który następnie wywoływany jest przez obiekty natywne w~maszynie wirtualnej. 