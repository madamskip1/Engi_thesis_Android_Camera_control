\newpage

\section{Detekcja mrugania}

Celem stwierdzenia czy wystąpiło mrugniecie zostało zaimplementowane proste rozwiązanie polegające na analizie określonej ilości klatek wstecz. Wykorzystywany jest fakt czy w~danej klatce oczy były zamknięte czy otwarte i~zmianę tego stanu system informuje o~ewentualnym wystąpieniu zdarzenia. 

\subsection{Algorytm detekcji mrugania}

\begin{enumerate}
    \item Utworzenie tablicy $T$ przechowującej $a$ wartości typu prawda/fałsz oznaczających czy oczy były zamknięte czy otwarte w~$a$ ostatnich klatkach obrazu.
    \item Usunięcie najstarszej wartości z~tablicy $T$ jeśli jest pełna
    \item Dodanie do tablicy $T$ nowej flagi zamkniętych oczu w~danej klatce
    \item Analiza zawartości tabeli $T$. Jeśli wszystkie pola mają taką samą wartość to ustalenie stanu $S_{i}$ na tę wartość i~porównanie z~poprzednim stanem~$S_{i-1}$:
        \begin{itemize}
            \item Jeśli $S_{i}$ jest inny niż $S_{i-1}$ i~stan $S_{i}$ oznacza oczy zamknięte to wystąpiło mrugnięcie
        \end{itemize}
    \item Powrót do punktu 3
\end{enumerate}

\vspace{3mm}

\textit{Uwaga 1.} Parametr $a$ oznacza ile klatek z~rzędu musi występować dany stan oka by uznać go za wiarygodny (służy odrzuceniu pojedynczych błędnych wskazań i~szumów).

\par

\textit{Uwaga 2.} Stan o~wartości prawda oznacza oczy zamknięte.



\subsection{Testowania detekcji mrugania na obrazie na żywo z~kamery} \label{section:test_eye_blink}

Przeprowadzone zostały testy detekcji ruchu oczu na obrazie na żywo. w~każdych warunkach zostały wykonane trzy testy powtórzone dwukrotnie:

\begin{itemize}
    \item mrugnięcie lewym okiem
    \item mrugnięcie prawym okiem
    \item mrugnięcie oboma oczami na raz
\end{itemize}

Poszczególne czynności w~ramach jednego testu wykonane był 10 razy. 

\begin{table}[!h]
\label{tab:eye_blink_accuracy}
\centering
\caption{Skuteczność wykrywania mrugania na obrazie na żywo z kamery}
\begin{tabular}{|c|c|c|c|c|}
\hline
 \textbf{\begin{tabular}[c]{@{}c@{}}Warunki\end{tabular}} &
  \begin{tabular}[c]{@{}c@{}}1.\end{tabular} &
  \begin{tabular}[c]{@{}c@{}}2.\end{tabular} &
  \begin{tabular}[c]{@{}c@{}}3.\end{tabular} &
  \begin{tabular}[c]{@{}c@{}}4.\end{tabular} \\ \hline \hline
\textbf{Lewym okiem} &
  9 &
  10 &
  10 &
  9 \\ \hline

\textbf{Prawym okiem} &
  10 &
  10 &
  8.5 &
  7 \\ \hline
  
\textbf{Oboma oczami} &
  10 &
  10 &
  10 &
  10 \\ \hline
  
  \hline
\end{tabular}%

\end{table}

Skuteczność detekcji mrugania okazała się bardzo dobra, lepsza niż oczekiwana. w~przypadku testu gdy czynność została wykonana oboma oczami na raz została wykryta ona za każdym razem. w~pozostałych przypadkach osiągała wynik na poziomie średnio~$92\%$ skuteczności. Takie rezultaty wydają się być całkowicie wystarczające i~zadowalające z~perspektywy pracy dyplomowej. 