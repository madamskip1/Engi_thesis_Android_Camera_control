\newpage

\section{Detekcja ruchu gałek ocznych}

Do detekcji ruchu gałek ocznych wykorzystywany jest identyczny mechanizm jak w~przypadku wykrywania mrugania. Analizowane są położenia źrenic w~określonej ilości klatek obrazu, a~na~ich podstawie zgłaszana jest ewentualna zmiana położenia oczu, a~w~konsekwencji ich ruch. 

\subsection{Algorytm detekcji ruchu oczu}

\begin{enumerate}
    \item Utworzenie tablicy $T$ przechowującej $a$ wartości oznaczających położenie oka w~$a$~ostatnich klatkach obrazu.
    \item Usunięcie najstarszej wartości z tablicy $T$ jeśli jest pełna
    \item Dodanie do tablicy $T$ nowego położenia oka danej klatce
    \item Analiza zawartości tabeli $T$  i porównanie z poprzednim stanem $S_{i-1}$:
    
    \begin{enumerate}
        \item Jeśli wszystkie pola mają taką samą wartość to ustalenie stanu $S_{i}$ na tę wartość. Inaczej skok do punktu 6
        \item Jeśli stan $S_{i}$ jest inny niż $S_{i-1}$ to:
        
        \begin{enumerate}
            \item Jeśli stan $S_{i-1}$ oznaczał oczy zamknięte to nastąpiło ich otwarcie
            \item Jeśli stan $S_{i-1}$ oznaczał położenie otwartych oczu to $S_{i}$ oznacza ich nowe położenie 
            \item Jeśli stan $S_{i}$ oznacza oczy zamknięte to nastąpiło ich zamknięcie
        \end{enumerate}
    \end{enumerate}
    
    \item Powrót do punktu 3
\end{enumerate}

\vspace{3mm}

\textit{Uwaga 1.} Parametr $a$ oznacza ile klatek z~rzędu musi występować dany stan oka by uznać go za wiarygodny (służy odrzuceniu pojedynczych błędnych wskazań i~szumów).

\par

\textit{Uwaga 2.} Stan może przyjmować wartości: lewo, środek, prawo oraz zamknięte. Pierwsze trzy oznaczają położenie tęczówki i~źrenicy (kierunek patrzenia użytkownika). 




\subsection{Testowanie detekcji ruchu oczu na obrazie na żywo z kamery}

Sposób badania był identyczny jak w~przypadku detekcji mrugania (rozdział~\ref{section:test_eye_blink}). Testowane były zdarzenia ruchu oczu w~lewo, w~prawo i~na przemian.

\begin{table}[!h]
\label{tab:eye_move_accuracy}
\centering
\caption{Skuteczność wykrywania ruchu oczu na obrazie na żywo z kamery}
\begin{tabular}{|c|c|c|c|c|}
\hline
 \textbf{\begin{tabular}[c]{@{}c@{}}Warunki\end{tabular}} &
  \begin{tabular}[c]{@{}c@{}}1.\end{tabular} &
  \begin{tabular}[c]{@{}c@{}}2.\end{tabular} &
  \begin{tabular}[c]{@{}c@{}}3.\end{tabular} &
  \begin{tabular}[c]{@{}c@{}}4.\end{tabular} \\ \hline \hline
\textbf{W lewo} &
  10 &
  10 &
  10 &
  8.5 \\ \hline

\textbf{W prawo} &
  9.5 &
  9.5 &
  10 &
  9 \\ \hline
  
\textbf{Na przemian} &
  10 &
  10 &
  10 &
  9.5 \\ \hline
  
  \hline
\end{tabular}%

\end{table}

W każdym teście wykrytych zostało co najmniej 8 na 10 zdarzeń. Aż w~14 przypadkach osiągając $100\%$ wykrywalności. Sumarycznie wykrywalność uplasowała się na poziomie średnio $96,6\%$. Podobnie jak w~detekcji mrugania rezultat ten wydaj się całkowicie adekwatny na potrzeby pracy dyplomowej i~projektu aplikacji. 