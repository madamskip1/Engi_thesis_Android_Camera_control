\newpage 

\section{Sposób badania algorytmów}

Algorytmy były badne i porównywane przy pomocy statycznych zdjęć z przygotowanego zbioru danych, jak i wykorzystując obraz na żywo z przedniej kamery urządzenia.

\subsection{Zbiór danych}
\label{section:dataset}
Na potrzeby badań i porównania algorytmów przygotowany został zbiór danych składający się~z~80 zdjęć zawierających twarze. 
\par
Źródła zdjęć:

\begin{itemize}
    \item 50 zdjęć wybranych z datasetu \textit{Young and Old Images Dataset} \cite{young_old_dataset}
    \item 30 zdjęć wykonanych przez autora pracy dyplomowej
\end{itemize}

Został on przygotowany w taki sposób, żeby zawierał zróżnicowane zdjęcia pod~wieloma względami, takimi jak: jakość obrazu, oświetlenie, powierzchnia zajmowana przez twarz, kolorystyka, płeć, kolor skóry, częściowe zakrycie twarzy, okulary czy odwrócona w bok głowa. 

\par

Wszystkie 80 zdjęć zostało opisanych ręcznie przez autora projektu na potrzeby badań, w~szczególności: obszar twarzy, oczu czy środek źrenic. 

\par

\subsection{Urządzenie do testów}

Wszystkie testy zostały przeprowadzone na urządzeniu Huawei P30 Pro w normalnym trybie wydajności i bez włączonego oszczędzania energii. 