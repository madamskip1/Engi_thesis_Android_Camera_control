\newpage 

\section{Sposób badania algorytmów}

Algorytmy będę badał i porównywał przy pomocy zarówno statycznych zdjęć z przygotowanego datasetu, jak i korzystając z obrazu na żywo z przedniej kamery urządzenia.

\subsection{Dataset}
\label{section:dataset}
Na potrzeby badań i porównania algorytmów przygotowałem dataset składający się~z~80 zdjęć zawierających twarze. 
\par
Źródła zdjęć:

\begin{itemize}
    \item 50 zdjęć wybranych z datasetu \textit{Young and Old Images Dataset} \cite{young_old_dataset}
    \item 30 zdjęć mojego autorstwa
\end{itemize}

Dataset został przygotowany w taki sposób, żeby zawierał zróżnicowane zdjęcia pod~wieloma względami, takimi jak: jakość obrazu, oświetlenie, powierzchnia zajmowana przez twarz, kolorystyka, płeć, kolor skóry, częściowe zakrycie twarzy, okulary czy odwrócona w bok głowa. 
Wszystkie 80 zdjęć zostało opisanych przeze mnie na potrzeby badań, w~szczególności: obszar twarzy, oczu czy środek źrenic. 
\par
Dodatkowo do badania detekcji źrenic zostanie użyty \textit{MRL Eye Dataset} \cite{mrl_eye_dataset} zawierający zdjęcia oczu wraz z pozycją środka źrenicy.

\subsection{Urządzenie do testów}

Wszystkie testy przeprowadzę będą na urządzeniu Huawei P30 Pro w normalnym trybie wydajności i bez włączonego oszczędzania energii. 