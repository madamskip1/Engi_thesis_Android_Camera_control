\begin{table}[!h]

\centering
\caption{Skuteczność algorytmu PF zależnie od zastosowanej funkcji projekcji na zbiorze danych.}
\label{tab:eye_pupil_projection_accuracy_result}
\begin{tabular}{|c|c|c|c|c|c|}
\hline
 &
  \textbf{\begin{tabular}[c]{@{}c@{}}W obszarze \\ tęczówki \end{tabular}} &
  \textbf{\begin{tabular}[c]{@{}c@{}}Poza obszarem \\ tęczówki \end{tabular}} &
  \textbf{\begin{tabular}[c]{@{}c@{}}Średni \\ błąd\end{tabular}} &
  \textbf{\begin{tabular}[c]{@{}c@{}}Błąd \\ <= 0.1\end{tabular}} &
  \textbf{\begin{tabular}[c]{@{}c@{}}Błąd \\ <= 0.5\end{tabular}}
  \\ \hline \hline

\textbf{Całkowa} &
  85 &
  51 &
  14,35\% &
  53 &
  26  \\ \hline
  
\textbf{Wariancji, kwadratowa} &
  98 &
  38 &
  13,51\% &
  60 &
  27 
  \\ \hline
  
\textbf{Wariancji, pierwiastkowa} &
  79 &
  57 &
  14,66\% &
  47 &
  29  \\ \hline
  
  \textbf{Wariancji, liniowa} &
  51 &
  85 &
  23,12\% &
  24 &
  4  \\ \hline
  
  \textbf{Ogólna, kwadratowa} &
  98 &
  38 &
  13,51\% &
  60 &
  27 
  \\ \hline
  
\textbf{Ogólna, pierwiastkowa} &
  85 &
  51 &
  14,27\% &
  53 &
  26  \\ \hline
  
  \textbf{Ogólna, liniowa} &
  86 &
  50 &
  14,17\% &
  53 &
  26  \\ \hline
  
  \hline
\end{tabular}%
\end{table}