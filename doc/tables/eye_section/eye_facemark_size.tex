\begin{table}[!h]
\label{tab:eye_facemark_size_result}
\centering
\caption{Skuteczność algorytmu detekcji oczu wykorzystując znaczniki twarzy zależnie od wielkości zwracanego obszaru na zbiorze danych.}
\resizebox{\textwidth}{!}{%
\begin{tabular}{|c|c|c|c|c|c|c|}
\hline
 &
  \textbf{\begin{tabular}[c]{@{}c@{}}Prawidłowe\\ detekcje\end{tabular}} &
  \textbf{\begin{tabular}[c]{@{}c@{}}Perfekcyjne\\ detekcje\end{tabular}} &
  \textbf{\begin{tabular}[c]{@{}c@{}}Częściowo\\ dobre\\ detekcje\end{tabular}} &
  \textbf{\begin{tabular}[c]{@{}c@{}}Złe\\ detekcje\end{tabular}} &
  \textbf{\begin{tabular}[c]{@{}c@{}}Niewykryte\\  oczy otwarte\end{tabular}}  &
 \textbf{\begin{tabular}[c]{@{}c@{}}Niewykryte\\  oczy zamknięte \end{tabular}} \\ \hline \hline
\textbf{\begin{tabular}[c]{@{}c@{}}Znaczniki oczu \\ bez zwiększenia obszaru\end{tabular}} &
  142 &
  17 &
  125 &
  14 &
  6 &
  6  \\ \hline

\textbf{\begin{tabular}[c]{@{}c@{}}Znaczniki oczu \\ ze zwiększeniem obszaru\end{tabular}} &
  144 &
  142 &
  2 &
  12 &
  6 &
  6  \\ \hline
  
  \hline
\end{tabular}%
}
\end{table}