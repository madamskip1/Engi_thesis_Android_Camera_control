\begin{table}[!h]
\label{tab:eye_detection_accuracy_result}
\centering
\caption{Skuteczność algorytmów detekcji oczu na zbiorze danych}
\resizebox{\textwidth}{!}{%
\begin{tabular}{|c|c|c|c|c|c|c|}
\hline
 &
  \textbf{\begin{tabular}[c]{@{}c@{}}Prawidłowe\\ detekcje\end{tabular}} &
  \textbf{\begin{tabular}[c]{@{}c@{}}Perfekcyjne\\ detekcje\end{tabular}} &
  \textbf{\begin{tabular}[c]{@{}c@{}}Częściowo\\ dobre\\ detekcje\end{tabular}} &
  \textbf{\begin{tabular}[c]{@{}c@{}}Złe\\ detekcje\end{tabular}} &
  \textbf{\begin{tabular}[c]{@{}c@{}}Niewykryte\\  oczy otwarte\end{tabular}}  &
 \textbf{\begin{tabular}[c]{@{}c@{}}Niewykryte\\  oczy zamknięte \end{tabular}} \\ \hline \hline

\textbf{Znaczniki oczu RGB} &
  144 &
  142 &
  2 &
  12 &
  6 &
  6  \\ \hline
  
\textbf{\begin{tabular}[c]{@{}c@{}}Znaczniki oczu \\ sk. szaro. \end{tabular}} &
  145 &
  140 &
  5 &
  11 &
  1 &
  6  \\ \hline
  
\textbf{Haar RGB} &
  133 &
  95 &
  38 &
  23 &
  9 &
  11  \\ \hline
  
\textbf{Haar sk. szaro.} &
  130 &
  95 &
  35 &
  26 &
  11 &
  7  \\ \hline
  
  \hline
\end{tabular}%
}
\end{table}