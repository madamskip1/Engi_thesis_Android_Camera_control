\begin{table}[!h]
\label{tab:face_filter_test}
\centering
\caption{Wynik porównania sposobów filtrowania detekcji}
\resizebox{\textwidth}{!}{%
\begin{tabular}{|c|c|c|c|c|c|c|}
\hline
 &
  \textbf{\begin{tabular}[c]{@{}c@{}}Prawidłowe\\ detekcje\end{tabular}} &
  \textbf{\begin{tabular}[c]{@{}c@{}}Perfekcyjne\\ detekcje\end{tabular}} &
  \textbf{\begin{tabular}[c]{@{}c@{}}Częściowo\\ dobre\\ detekcje\end{tabular}} &
  \textbf{\begin{tabular}[c]{@{}c@{}}3/4\\ krawędzie\\ perfekcyjne\end{tabular}} &
  \textbf{\begin{tabular}[c]{@{}c@{}}Złe\\ detekcje\end{tabular}} &
  \textbf{\begin{tabular}[c]{@{}c@{}}Niewykryte\\ twarze\end{tabular}}  \\ \hline \hline
\textbf{Autorskie filtrowanie 300x300} &
  80 &
  62 &
  18 &
  18 &
  0 &
  0  \\ \hline
  
\textbf{Autorskie filtrowanie  500x500} &
  80 &
  68 &
  12 &
  11 &
  0 &
  0  \\ \hline
  
\textbf{Najwyższy współczynnik pewności 300x300} &
  76 &
  58 &
  18 &
  18 &
  4 &
  4  \\ \hline

  \textbf{Najwyższy współczynnik pewności 500x500} &
  76 &
  64 &
  12 &
  11 &
  4 &
  4  \\ \hline
 
  \hline
\end{tabular}%
}
\end{table}